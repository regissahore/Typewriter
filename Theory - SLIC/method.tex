% !Mode:: "TeX:UTF-8"
\documentclass[11pt, a4paper]{article}
\usepackage{CJK}
\usepackage[CJKbookmarks=true]{hyperref}
\usepackage{graphicx}
\usepackage{algorithm}
\usepackage{algorithmic}
\renewcommand{\algorithmicrequire}{\textbf{Input:}}
\renewcommand{\algorithmicensure}{\textbf{Output:}}
\begin{document}
\begin{CJK}{UTF8}{song}
\title{光教练大讲堂 - CLIC图像分割}
\author{光教练}
\maketitle
\section{目的}
介绍\cite{Bib_SLIC}中提到的图像分割的方法。
\section{方法}
算法所需要的参数只有一个$k$,为图像中分块的数量。$k$越大分块越细致,反之亦然。使用类似K-Means的方法,最初点在图中均匀选择,选好之后需要选择$3 \ times 3$ 区域中梯度最小的方向,防止对于初始中心点选到边上。对于彩色图像,我们需要分块的中心点的特征为CIELAB 的颜色和位置信息,即$C_i=[l_i\: a_i\: b_i\: x_i\: y_i]^T$,分块的边长大致为$S=\sqrt{N/K}$ 个像素。规定距离度量,如果点到当前中心点的距离大于$D$,就可以不做考虑,这样可以加快运算速度,通常区域大约为$S \times S$,考虑区域$2S \times 2S$即可。接下来用和K-Means 近似的方法迭代即可。定义剩余偏差$E$为两次聚类中心位置的变化,当$E$小于一个阈值时即可退出。最后需要将不相连的像素分配给最近的中心。
由于既存在颜色信息又存在位置信息,需要重新定义距离的度量。令$N_s$和$N_c$分别为最大空间距离和颜色距离,定义距离:
\begin{equation}
d_c=\sqrt{(l_j-l_i) ^ 2 + (a_j-a_i) ^ 2 + (b_j-b_i) ^ 2}
\end{equation}
\begin{equation}
d_s=\sqrt{(x_j-x_i) ^ 2 + (y_j-y_i) ^ 2}
\end{equation}
\begin{equation}\label{eq_dd}
D'=\sqrt{(\frac{d_c}{N_c}) ^ 2 + (\frac{d_s}{N_s}) ^ 2}
\end{equation}
而最大空间距离是已知的,为$N_s = S = \sqrt{N/K}$。同时可以将颜色距离最大值定义为常量m,所以式\ref{eq_dd}可以更改为:
\begin{equation}
D'=\sqrt{(\frac{d_c}{m}) ^ 2 + (\frac{d_s}{S}) ^ 2}
\end{equation}
可以使用
\begin{equation}
D'=\sqrt{({d_c} ^ 2 + (\frac{d_s}{S}) ^ 2 m^2}
\end{equation}
作为距离的度量。最后的算法如\ref{alg:SLIC}所示。
\begin{algorithm}
\caption{SLIC分割算法}
\label{alg:SLIC}
\begin{algorithmic}
\STATE 以步长$S$初始化聚类中心$C_k=[l_k\: a_k\: b_k\: x_k\: y_k]^T$。
\STATE 移动聚类中心到$3 \times 3$区域梯度最低的点上。
\STATE 为每个像素设置标签$l(i)=-1$
\STATE 为每个像素设置距离$d(i)=\infty$
\REPEAT
\FOR{对于每个聚类中心$C_k$}
\FOR{对于$C_k$周围$2S \times 2S$区域的所有像素$i$}
\STATE 计算$C_k$和$i$的距离$D$
\IF{$D<d(i)$}
\STATE $d(i) = D$
\STATE $l(i) = k$
\ENDIF
\ENDFOR
\ENDFOR
\STATE 计算新的聚类中心
\STATE 计算剩余偏差$E$
\UNTIL $E$小于等于一个阈值
\end{algorithmic}
\end{algorithm}
最后找孤立点可以用聚类中心的每一个点做一遍深搜并标记,最后没被标记的就是孤立点。梯度可以用$G(x,y)=\sqrt((l(i + 1, j) - l(i, j)) ^ 2 + (l(i, j + 1) - l(i, j)) ^ 2)$。
\begin{thebibliography}{0}
	\bibitem {Bib_SLIC} Achanta R, Shaji A, Smith K, et al. SLIC superpixels compared to state-of-the-art superpixel methods[J]. 2012.
\end{thebibliography}
\end{CJK}
\end{document}
