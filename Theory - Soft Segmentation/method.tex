% !Mode:: "TeX:UTF-8"
\documentclass[11pt, a4paper]{article}
\usepackage{CJK}
\usepackage[CJKbookmarks=true]{hyperref}
\usepackage{graphicx}
\usepackage{algorithm}
\usepackage{algorithmic}
\renewcommand{\algorithmicrequire}{\textbf{Input:}}
\renewcommand{\algorithmicensure}{\textbf{Output:}}
\begin{document}
\begin{CJK}{UTF8}{song}
\title{光教练大讲堂 - 软分割}
\author{光教练}
\maketitle
\section{目的}
\section{框架}
\section{方法}
\subsection{过分割}
\subsubsection{建图}
论文\cite{Bib_Segment}中提到的过分割算法是基于图结构的,通过不断将近似的区域合并最终得到分割的图像。一种建图方法为,图像中所有像素都是图的节点,每个像素与周围8个点连边,边权为特征向量的距离。另外一种方法的节点定义不变,但边为两个像素特征向量小于一个阈值时两个像素之间有边,不考虑像素在图像中的位置。
\subsubsection{算法}
\begin{algorithm}
\caption{过分割算法}
\label{alg:OverSeg}
\begin{algorithmic}[1]
\REQUIRE
有$n$个节点和$m$条边的图$G=(V, E)$,边权为$\omega$。
\ENSURE
$r$个组件的分割$S=(C_1,...,C_r)$,每个集合$C$包含部分节点。
\STATE 将边$E$按边权由小到大排序为$\pi=(o_1, ..., o_m)$。
\STATE 初始化分割$S^0$,每个节点都属于自身的集合。
\FOR {$q=1,...,m$}
\STATE 如果$i$和$j$是边$o_q$连接的两个节点,如果$C_{i}^{q-1} \neq C_{j}^{q-1}$ 且边权$\omega(o_q)<MInt(C_i^{q-1}, C_j^{q-1})$,则将节点$i$和$j$合并到同一集合中。
\ENDFOR
\RETURN $S=S^m$
\end{algorithmic}
\end{algorithm}
算法\ref{alg:OverSeg}中,$MInt(C_i^{q-1}, C_j^{q-1})$为一个集合的内部差异,$Int(C_i^{q-1}, C_j^{q-1})=min(Int(C_i)+\tau(C_i), Int(C_j)+\tau(C_j))$,其中$\tau(C)=k/|C|$,$|C|$为当前集合中点的数目。$\tau(C)$是一个阈值函数,为了防止点比较少时内部差异也小导致难以合并集合,$k$是一个常数,可以取值100。$Int(C)$初值可以取0,每次合并集合时,$Int(C)$更新为$max(Int(C_i), Int(C_j), \omega(o_q))$。
\subsection{置信度传播}
\subsection{软分割}
\begin{thebibliography}{0}
    \bibitem {Bib_Enhance} Wang B, Yu Y, Wong T T, et al. Data-driven image color theme enhancement[C]//ACM Transactions on Graphics (TOG). ACM, 2010, 29(6): 146.
	\bibitem {Bib_AppProp} An X, Pellacini F. AppProp: all-pairs appearance-space edit propagation[C]//ACM Transactions on Graphics (TOG). ACM, 2008, 27(3): 40.
	\bibitem {Bib_Segment} Felzenszwalb P F, Huttenlocher D P. Efficient graph-based image segmentation[J]. International Journal of Computer Vision, 2004, 59(2): 167-181.
	\bibitem {Bib_BP} Yedidia J S, Freeman W T, Weiss Y. Understanding belief propagation and its generalizations[J]. Exploring artificial intelligence in the new millennium, 2003, 8: 236-239.
\end{thebibliography}
\end{CJK}
\end{document}
